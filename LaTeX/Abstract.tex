\documentclass[12pt,a4paper]{report}
\usepackage[utf8]{inputenc}
\usepackage{setspace}
\usepackage{titlesec}
\usepackage{fancyhdr} 
\usepackage{lipsum} 
\usepackage{amsmath} % For mathematical formatting
\usepackage{graphicx}
\usepackage{color}
\usepackage{listings}
\usepackage{newtxtext} % Times New Roman for text
\usepackage{newtxmath} % Times New Roman for math
\renewcommand{\rmdefault}{ptm} % Set Times New Roman as the default font
\renewcommand{\sfdefault}{phv} % Helvetica for sans-serif
\renewcommand{\ttdefault}{pcr} % Courier for typewriter font
\titleformat{\chapter}[hang]{\bfseries\Large}{\thechapter.}{1em}{}
\titleformat{\section}[hang]{\bfseries\large}{\thesection}{1em}{}
%\usepackage{hyperref}
\usepackage{xcolor, colortbl}
\setlength{\topmargin}{-0.5in}
\setlength{\oddsidemargin}{0.5in}
\setlength{\evensidemargin}{0.5in}
\setlength{\textwidth}{6in}
\setlength{\textheight}{9in}
\setstretch{1.5}
\usepackage{pgfplots}
\usepackage{acronym}
\usepackage{pgfplots}
\pgfplotsset{compat=1.17}

\definecolor{codegray}{rgb}{0.5,0.5,0.5}
\definecolor{codepurple}{rgb}{0.58,0,0.82}
\definecolor{backcolour}{rgb}{0.95,0.95,0.92}
\definecolor{lightblue}{rgb}{0.678, 0.847, 0.902}
\definecolor{lightpink}{rgb}{1.0, 0.71, 0.76}
\definecolor{lightcoral}{rgb}{0.941, 0.502, 0.502}
\definecolor{peachpuff}{rgb}{1.0, 0.99, 0.99}

\lstdefinestyle{mystyle}{
	backgroundcolor=\color{peachpuff},
	commentstyle=\color{codegray},
	keywordstyle=\color{blue},
	numberstyle=\tiny\color{blue},
	stringstyle=\color{codepurple},
	basicstyle=\ttfamily\footnotesize,
	breakatwhitespace=false,
	breaklines=true,
	captionpos=b,
	keepspaces=true,
	numbers=left,
	numbersep=5pt,
	showspaces=false,
	showstringspaces=false,
	showtabs=false,
	tabsize=2
}
\lstset{style=mystyle}

\pagestyle{fancy}
\fancyhf{}
\fancyhead[R]{\thepage}
\fancyfoot{}
\renewcommand{\headrulewidth}{0pt} 

\begin{document}
	
\pagenumbering{roman}  % Switch to Roman page numbering
\setcounter{page}{1} 

	




















	
   \begin{center}
	\textbf{\Large ABSTRACT}\\[0.5cm]
\end{center}
	
Tomato quality grading plays a crucial role in ensuring high standards in agricultural production. Traditional manual grading methods are often inefficient, labor-intensive, and prone to errors, resulting in increased operational costs and reduced quality control. This research proposes an automated tomato quality grading system that integrates pre-trained convolutional neural networks (CNNs) for feature extraction with traditional machine learning classifiers for classification. The system utilizes both binary and multi-class classification techniques to assess tomato quality, categorizing them into healthy or rejected (binary), and ripe, unripe, old, and damaged (multi-class).

 The hybrid approach leverages CNNs such as InceptionV3, DenseNet121, and MobileNetV2 for extracting intricate features from tomato images, which are then classified using machine learning models like Support Vector Machines (SVM), k-Nearest Neighbors (KNN), and Decision Trees (DT). Experimental results show that the combination of DenseNet121 with a Support Vector Classifier (SVC) and a linear kernel achieves 96\% accuracy in multi-class classification, while InceptionV3 with an SVC and an RBF kernel reaches 94\% accuracy in binary classification. This study demonstrates the potential of deep learning and machine learning hybrid models in automating and enhancing the accuracy, efficiency, and scalability of tomato quality grading, ultimately offering a cost-effective and reliable solution for modern agricultural practices.
 
 

 \newpage
 \setcounter{page}{3} 
 \begin{center}
 	\textbf{\Large ACKNOWLEDGEMENT}\\[0.4cm]
 \end{center}
 
 I would like to express my humble gratitude and heartfelt thanks to my project supervisor, \textbf{Dr. S. Anbuchelian}, Associate Professor and Deputy Director, Ramanujan Computing Centre, College of Engineering -- Guindy, Anna University, for his valuable time, encouragement, guidance, and advice throughout this study.
 
 \vspace{1em}
 
 I sincerely thank \textbf{Dr. V. MARY ANITA RAJAM}, Professor and Head, Department of Computer Science \& Engineering, College of Engineering -- Guindy, Anna University, for the opportunity and facilities provided for this project.
 
 \vspace{1em}
 
 I extend my gratitude to \textbf{Dr. R. S. Bhuvaneswaran}, Professor \& Director, Ramanujan Computing Centre, College of Engineering -- Guindy, Anna University, for his support and for providing the necessary resources for this work.
 
 \vspace{1em}
 
 I am deeply grateful to \textbf{Dr. A. John Prakash}, Associate Professor, Deputy Director, and Panel Member, Ramanujan Computing Centre, College of Engineering -- Guindy, Anna University, for his guidance and support during my thesis work.
 
 \vspace{1em}
 
 I also extend my sincere thanks to \textbf{Dr. S. Lokesh}, Associate Professor and Project Coordinator, Ramanujan Computing Centre, College of Engineering -- Guindy, Anna University, for his valuable suggestions for my thesis.
 
 \vspace{1em}
 
 I am thankful to the course-handling staff, including \textbf{Dr.K.Murugan} Professor
 Former Director, \textbf{Dr. T. P. Saravanabava}, Program In-Charge and Professor, \textbf{Dr. D. George Washington}, \textbf{Dr. H. Khanna Nehemiah}, Professors, and \textbf{Dr. R. Shathanaa}, Assistant Professor, of the Ramanujan Computing Centre, College of Engineering -- Guindy, for their guidance and support.
 
 \vspace{1em}
 
 Finally, I would like to thank \textbf{Ms. Aishwarya}, Research Scholar at Ramanujan Computing Centre, for her assistance during this project.
 
 
 
 \begin{flushright}
 	\textbf{RAGUPATHI M} \\
 	(2023184032)
 \end{flushright}

\end{document}

